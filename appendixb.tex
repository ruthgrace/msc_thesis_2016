\chapter{NAFLD study data collection}\label{AppB}
\myappendices{Appendix ~\ref{AppB} ~\nameref{AppB}}

The entire contents of Appendix ~\ref{AppB} were written by Hannah Da Silva from Allard research group in Toronto, and provided through personal correspondence.

\subsection{Study Participants}
This cross-sectional study includes 39 NAFLD patients (15 SS, 24 NASH) from the University Health Network (UHN) outpatient liver clinics and 30 healthy living liver donors as healthy controls (HC) from the UHN Liver Transplant Clinic. Patient recruitment occurred from July 2010 to May 2014. The University Health Network and University of Toronto Research Ethics Boards approved this research. All subjects provided their informed written consent.

Patients suspected of having NAFLD due to persistently elevated liver enzymes after ruling out other causes were recruited at their hepatologist appointment prior to liver biopsy. Following a detailed explanation of the study and consent form, and answering of any questions, informed consent was obtained. Eligibility to participate in the study was then considered. For NAFLD patients the inclusion criteria were: a biopsy proven diagnosis of NAFLD (determined during the study), age greater than or equal to18 years, alcohol consumption of less than 20g per day. NAFLD patients were excluded from the study if they had a diagnosis of any other liver disease or HIV infection, if liver transplant was expected to be required within one year, if they had significant liver complications (e.g. variceal bleeding, jaundice, etc.) or any contraindications for liver biopsy, if they were pregnant or lactating, if they had any gastrointestinal disease, or if they were taking any medications known to cause steatohepititis, insulin, NSAIDS, antibiotics, prebiotics, probiotics, or experimental drugs within the last three months. Healthy living liver donors were approached at their first screening appointment for liver donation. Upon completion of informed consent healthy donors were screened for research eligibility. Inclusion and exclusion criteria were the same with the additional inclusion criteria that they must be eligible for live liver donation.

\subsection{Study Visits}
Each participant attended three study visits as outlined in Figure 1. For NAFLD patients, after informed consent was obtained, they were provided with detailed instructions for the completion of a 7-day food record, 7-day activity record, and environmental questionnaire (see below for further details). Patients were also provided with instructions for the collection and transport of their stool sample, which they were requested to return on the day of their scheduled liver biopsy. Usual clinic blood work was also collected at this initial visit for general markers of liver health. On their second visit fasting study-specific blood work was collected and anthropometric measurements were completed. The third visit was the day of their liver biopsy when patients also returned their stool sample and food and activity logs. Healthy living liver donors followed a similar schedule, however, stool samples were typically collected on visit two and the liver biopsy was taken intraoperatively. For further details on all study measurements and processing see below.

\subsection{Clinical Data, Environmental Questionnaire, and Anthropometric Measurements}
Clinical data was collected on study visit one. Study participant’s smoking and alcohol consumption history and medication and supplement use were reviewed, including medications taken in the last three months. Study participants were also asked to answer a number of questions regarding their personal and family history of disease. Age, ethnicity, and menstrual history were also recorded.

An environmental questionnaire was completed and returned with the stool sample. This questionnaire collected information that may affect an individual’s IM composition including country of origin, method of birth (vaginal versus caesarian section), whether they were breastfed as an infant, what kind of pets they have at home, and others.

Anthropometric measurements including height (ht), weight (wt), waist circumference (WC), hipcircumference, and weight-to-hip ratio (WHR) were measured by a trained research professional. Weight was measured using a calibrated hospital-grade chair scale; height was measured using a standometer. Waist circumference was measured at the umbilicus level and hip circumference was measured at the widest point over the buttock. All measurements were taken in triplicate and the average value was used.

\subsection{Nutrition and Activity Assessment}
Each participant was given a food record and activity log to complete in the weekprior to returning their stool sample. Detailed instructions were provided for the completion of both of these tools. The food log included all food and beverages consumed each 24 hours for seven days. In cases where time was insufficient a three-day food record including one weekend day was completed.  Participants used the 2D Food Portion Visual Chart (Nutrition Consulting Enterprises, Framingham, MA) to estimate portion sizes. This is a validated tool which has been used in our previous studies [59, 187, 188]. Food records were reviewed by an experienced registered dietitian and were analyzed using Food Processor Diet and Nutrition Analysis Software (Version 7, ESHA Research, Salem, OR).

Physical activity logs were recorded for 7 days concurrent with the food records. Participants were asked to record any activity, including household chores, the duration of the activity, and the intensity level. Detailed instructions were provided including examples for each intensity level (mild, moderate, strenuous, and very strenuous).  This information was used to calculate daily physical activity units: 1 unit = 30 minutes mild, 20 minutes moderate, 10 minutes strenuous, or 5 minutes very strenuous activity. This is a validated method for measuring physical activity level [189]. Basal metabolic rate (BMR) was calculated using the Harris-Benedict equation: BMR for men = 66.5 + [13.75 × wt(kg)] + [5.003 × ht(cm)] – [6.755 × age(y)], BMR for women = 655.1 [9.563  × wt(kg)] + [1.850  × ht(cm)] – [4.676  × age(y)]. Estimated energy expenditure (EER) was calculated using Health Canada Guidelines: EER for men = 662 – [9.53 x age(y)] + PA x {[15.91 x wt(kg])] + [539.6 x ht(m)]}, and EER for women = 354 – [6.91 x age(y)] + PA x {[9.36 x wt(kg)] + (726 x ht(m)]} where PA is the physical activity coefficients.

\subsection{Biochemistry}
Routine and study specific blood work was drawn after a 12 hour overnight fast on study visittwo. Liver markers were drawn as a routine clinical measure and included: aspartate transaminase (AST), alanine transaminase (ALT), alkaline phosphatase (ALP), and bilirubin. Measures of glucose metabolism included plasma glucose, insulin, hemoglobin A1c (HbA1c), and HOMA-IR which was calculated as fasting glucose (mmol/L) × fasting insulin (mU/L)/22.5 [191]. A lipid profile was also conducted, including total cholesterol, low density lipoprotein (LDL), high density lipoprotein (HDL), and triglycerides (TG). These analyses were conducted by the UHN Laboratory Medicine Program. Liver enzymes and lipid profile were measured using the Architect c8000 system (Abbott Laboratories). LDL was calculated from total cholesterol – HDL. Fasting plasma glucose and plasma insulin were measured by the enzymatic hexokinase method and radioimmunoassay, respectively.

\subsection{Serum Metabolites}
Serum metabolites, including choline, ethanol, and TMA, were measured to evaluate potential implications of bacterial metabolism at a systemic level. For the full list of the 41 metabolites measured see Table 4. Serum was drawn in a fasting state using a gel serum separation vacutainer and was immediately placed in an insulated container with cooling elements. Blood was separated by centrifuge at 4°C at 2800 x g for 20 minutes.  Serum was then aliquoted and stored at -80°C until all study samples were collected. Serum was then shipped to the Metabolomic Innovation Centre (Edmonton, AB) were metabolites were analyzed using nuclear magnetic resonance  spectrometry (NMR), a method which this centre has perfected [192]. The following methods were used by the centre and are stated in their own words [192]: All serum samples were deproteinized using ultrafiltration.  Prior to filtration, two 0.5 mL, 3 KDa cut-off centrifugal filter units (Millipore Microcon YM-3) were rinsed four times each with 0.5 mL of water, then centrifuged at 11 000 rpm for 1 hour, to remove residual glycerol bound to the filter membranes. Two 150 μL aliquots of each serum sample were then transferred into the two centrifuge filter devices. The samples were then spun at a rate of 11 000 rpm for 140 minutes, to remove macromolecules (primarily proteins and lipoproteins) from the sample. The subsequent filtrates were then checked visually for a red tint, which indicates that the membrane was compromised. For those “membrane compromised” samples, we repeated the filtration process with a different filter and inspected the filtrate again.  We then pooled the filtrates that passed the inspections and recorded the volume. If the total volume of the sample was under 300 μL, we added an appropriate amount from a 50 mM NaH2PO4 buffer (pH 7) to the sample until the total volume was 300 μL.  Subsequently, 35 μL of D2O and 15 μL of a standard buffer solution [11.667 mM DSS (disodium-2,2-dimethyl-2-silapentane-5-sulphonate), 730 mM imidazole, and 0.47\% NaN3 in H2O] was added to the sample. The serum sample (350 μL) was then transferred to a standard Shigemi microcell NMR tube for subsequent spectral analysis.

All 1H-NMR spectra were collected on a 500 MHz Inova (Varian Inc., Palo Alto, CA) spectrometer equipped with either a 5 mm HCN Z-gradient pulsed-field gradient (PFG) room-temperature probe or a Z-gradient PFG Varian cold-probe. 1H-NMR spectra were acquired at 25°C using the first transient of the tnnoesy-presaturation pulse sequence, which was chosen for its high degree of quantitative accuracy [193]. Spectra were collected with 128 transients and 8 steady-state scans using a 4 second acquisition time and a 1 second recycle delay.

All FIDs were zero-filled to 64k data points and subjected to line broadening of 0.5 Hz. The singlet produced by a known quantity the DSS methyl groups was used as an internal standard for chemical shift referencing (set to 0 ppm) and for quantification. All 1H-NMR spectra were processed and analyzed using the Chenomx NMR Suite Professional software package version 6.0 (Chenomx Inc., Edmonton, AB), as previously described [194].  Each spectrum was processed and analyzed by at least two experienced NMR spectroscopists to minimize compound mis-identification and mis-quantification.

Serum trimethylamine N-oxide (TMAO) was not detectable using NMR therefore TMAO was measuring using TMAO using a targeted quantitative metabolomics approach by Liquid Chromatography Mass Spectrometry (LCMS). Isotopically-labeled internal standards were added to the serum to facilitate metabolite quantification. Sample extraction was performed on a 96 well plate with a 0.2 µm solvent filter. 10 µL of serum was spiked with the internal standard (TMAO D9) and then 150 µL of methanol with 10 mM ammonium acetate was added for extraction. The plate was shaken for 10 min and centrifuged at 500 rpm for 5 minutes at 4 ºC. Each sample was diluted with 150 μL of water. Seven calibrant solutions with known concentrations went through the same extraction steps. LCMS analysis was performed on AB SCIEX 4000 QTrap mass spectrometer with Agilent 1100 HPLC. 10 µL of the extracted samples were injected onto the Kinetex C18 (2.6 µm, 3.0x100mm, 100A) Column with guard column. Isobaric elution was performed with 90\% A (10 mM ammonium formate in water, PH3) and 10\% B (10 mM ammonium formate in 90:10 Acetonitrile:water, PH3). Total LC method run time was 3 min with flowrate of 500 µl/min. A seven-point calibration curve was generated to quantify the concentration of TMAO in samples.

\subsection{Liver Histology}
Liver biopsies were taken percutaneously (needle biopsy) for NAFLD patients and intraoperatively (wedge biopsy) for HC and preserved immediately in formalin. Liver biopsies were assessed by the same pathologist using standard stains for the diagnosis of NAFLD, morphologic evaluation, and to rule out any iron overload. The evaluation of NAFLD related measures of steatosis, inflammation, and fibrosis were conducted using the validated and reproducible Brunt system. Disease severity was also evaluated using the NAFLD Activity Score (NAS) which accounts for degree of steatosis, lobular inflammation, and hepatocellular ballooning for a final score of 0-8.

\subsection{Stool Sample Collection and Analysis}
On study visit one participants received a stool collection kit, including a plastic collection/storage container with a tightly closing lid, an insulated bag, and cooling elements. Within 24 hours of their next appointment they collected one stool sample, which was frozen immediately after defecation in the patient’s home freezer (-20°C). Participants brought the frozen sample in the insulated bag with cooling elements to their appointment at the hospital, where it will stored at -80C until homogenization.

\subsection{Stool Homogenization}
Stool samples were homogenized prior to DNA extraction for IM sequencing and metabolitemeasurements. The entire sample was first transferred into a sterile masticator bag. The sample was allowed to thaw until a smooth consistency was reached, typically 2-3 hours depending on sample size. Once thawed excess air was released from the bag and the sample was homogenized for two minutes using a masticator blender (IUL, S.A., Barcelona, Spain). This was followed by one minute of hand mastication of any areas that were missed. The corner of the masticator bag was then cut and the sample was aliquoted: 1-2 g samples were stored for metabolite analysis and 0.1-0.2 g aliquots were stored for DNA extraction. Samples were immediately placed on dry ice and then transferred to -80°C for storage until analysis. Weight was recorded for each aliquot and pH was measured for each sample.

\subsection{DNA Extraction}
DNA was extracted using the E.Z.N.A. Stool DNA Kit (Omega Bio-Tek, Norcross, GA) and amodified manufacturer’s protocol. Briefly, 200 mg of glass beads and 600 µL of SLB buffer were added to the sample and vortexed at maximum speed for 15 minutes. 20 µL of lysozyme (20mg/ µL) was added and flicked to mix then incubated at 37°C for 30 minutes. 60 µL DS Buffer and 20 µL Proteinase K were added and vortexed to mix then incubated at 70°C, 300 rpm, for 13 min, vortexing at T=6.5 minutes and T=13 minutes. The incubation temperature was then increased to 95°C for an additional 5 minutes. 200 µL SP2 Buffer was added and mixed by vortex for 30 seconds then put on ice for 5 minutes. The mixture was then centrifuged at 21 000 rcf for 7 minutes and the supernatant was transferred to a new 1.5 mL centrifuge tube while the old tube was discarded. 200 µL HTR Reagent was added and vortexed at maximum speed for 10 seconds, then incubated at room temperature for two minutes and centrifuged for an additional two minutes. The supernatant was again transferred to a new 1.5 mL centrifuge tube and the addition of HTR Reagent and following steps were repeated once. After transfer to the last 1.5 mL tube 250 µL of BL buffer and absolute ethanol were added and vortexed for 10 seconds.

The DNA column was placed into a collection tube and 100 µL of 3M NAOH was added. This was incubated for four minutes and centrifuged for one minute. 100 µL of distilled water was added to the DNA column and centrifuged for one minute. 800 µL of sample was transferred into the DNA column and centrifuged for one minute. The contents of collection tube was discarded and the remainder of the sample was transferred into the DNA column and again centrifuged for one minutes. The flow-through and collection tube were discarded. The column was then placed into a new 2 mL collection tube and 500 µL of VHB Buffer was added to the column. This was centrifuged for 30 seconds and the flow-through was discarded. 700 µL of DNA Wash Buffer was then added to the DNA column and centrifuged for one minute and the flow-through and tube was discarded. This washing stage was repeated once. The column was then transferred to a new 2 mL collection tube, centrifuged for one minute and the flow-through was discarded. The tube was then centrifuged again for two minute, this time with the cap open, to the dry the column. The column was finally transferred to a new 1.5 mL tube, 100 uL of distilled water was added, this was incubated for five minutes, centrifuged for two minutes, the column was removed and then the final DNA sample was analyzed for purity and concentration using the Nanodrop 1000 Spectrophotometer (ThermoScientific, Rockford, IL). DNA samples were stored at -80°C.

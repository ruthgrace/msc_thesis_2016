\chapter{Time series: Long memory}
Here is a picture of a long memory time series. 
\begin{figure}[ht]
\begin{center}
\includegraphics[height = 9cm, width = 9cm]{pic1.jpeg}
\caption{A long memory time series\label{ts1}}
\end{center}
\end{figure}

Here's a table.
\begin{table}[ht]
\begin{center}
\begin{tabular}[ht]{|c|lr|c|} 
%c stands for centre, l for left, r for right; the | puts lines in between, and the hline puts a horizontal line in
\hline
$n$ & $\alpha$ &$n\alpha$ & $\beta$\\
\hline
1 & 0.2 & 0.2 & 5\\
\hline
2 & 0.3 & 0.6 & 4\\
\hline
3 & 0.7 & 2.1 & 3\\
\hline
\end{tabular}
\caption{A random table \label{tab1}}
\end{center}
\end{table}

\begin{eqnarray}
y &=& mx + b \label{eq1}\\
&=& ax+ c
\label{eq2}
\end{eqnarray}

This is an un-numbered equation, along with a numbered one. 
\begin{eqnarray}
u &=& px \nonumber\\
p &=& P(X=x) \label{eqn3}
\end{eqnarray}

Look at Table \ref{tab1} and Figure \ref{ts1} and equations \ref{eq1},  \ref{eq2}, and \ref{eqn3}.

Let's do some matrix algebra now.

\begin{equation}
det\left(\left|\begin{array}{ccc} 2 & 3 & 5\\
4 & 4 & 6\\
9 & 8 & 1
\end{array}\right|\right) = 42
\end{equation}

In the equation and eqnarray environments, you don't need to have the dollar sign to enter math mode.

\begin{eqnarray}
\alpha = \beta_1 \Gamma^{-1}
\end{eqnarray}

This is citing a reference ~\cite{mygood11111}.  This is citing another ~\cite{mrx05}.  Nobody said something ~\cite{Nobody06}.

\chapter{Discussion}

\section{Rampant lack of reproducibility}
It is clear that more robust statistical analysis is necessary in this field. The chapter on expanding the UnIFrac toolbox highlighted examples of misuse of unweighted UniFrac in papers published in Cell and Nature. One claimed to find differences in the gut microbiome of mice modelling autism spectrum disorder, compared to healthy controls. The other claimed to find differences in the gut microbiome of humanized mice fed a more traditional fibrous diet compared to mice compared to mice fed a diet similar in composition to the Western diet. These studies had small sample sizes (n = 20 and 10 respectively), used unweighted UniFrac (which we have shown to be unreliable), and had a low amount of variance explained by the principal components axes (14\% on PC1).

The chapter about nonalcoholic fatty liver disease (NAFLD) showcased five studies which all claimed to have found a difference in the gut microbiome of patients with nonalcoholic fatty liver disease compared to healthy controls, but with almost non-overlapping results (Fig.~\ref{nafld_fig1}). Some of the variation can be explained by differences in sequencing platform (Roche 454 vs. Illumina MiSeq). More variation can be explained by the variable region of the 16S rRNA gene chosen for sequencing - one study used V1-2, one study used V3, one study used V4, and the other two studies did not report which variable region was used. Three out of five studies used healthy controls with a lower BMI than the NAFLD group, such that differences due to level of obesity could not be distinguished from differences due to NAFLD. Lastly, only one of the studies performed a multiple test correction, so most of the results could not be distinguished from false positives.

Recently the social sciences, particularly psychology, has come under fire for producing irreproducible results, to the point where some claim that most findings are actually false \cite{ioannidis2005most}. The biomedical sciences suffer similar issues, prompting Nature to publish a collection of statistics for biologists (\url{http://www.nature.com/collections/qghhqm/}).

\section{Recommendations}
Throughout this thesis we have made a case for compositional data analysis. Currently the analytical tools with the most widespread use in the field are the unweighted and weighted UniFrac distance for principal components analysis or principal coordinates analysis, along with software such as metagenomeSeq \cite{paulson2014metagenomeseq}, DESeq2 \cite{love2014moderated}, and metastats \cite{paulson2011metastats} for differential expression analysis. These are commonly accessed through pipelines such as QIIME and mothur. Many of these have roots in ecology, for example, the Shannon diversity index. Shannon diversity is commonly measured in microbiome papers, but does not make sense for complex biological samples where diversity can be increased by performing deeper sequencing to uncover more bacterial taxa.

While compositional data analysis may not be at a stage where it is ready to set as the standard analytical tool, we believe that this model is much closer to the correct answer than the standard toolkit used by microbiome researchers. Recommended compositional data resources include the book \textit{Analyzing compositional data with R} \cite{van2013analyzing}, the 16S rRNA gene sequencing compositional analysis workshop (hosted online \href{https://github.com/ggloor/compositions/blob/master/background_reading/CJM_supplement/workshop.pdf}{on GitHub}) and all the other resources hosted by the CoDa organization (\url{http://www.compositionaldata.com/}). Recommended software and tools for microbial network correations include SPARCC \cite{friedman2012inferring}, SpiecEasi \cite{kurtz2015sparse} or the phi metric \cite{lovell2015proportionality}. Recommended software and tools for differential expression analysis include the analysis of composition of microbiomes (ANCOM) \cite{mandal2015analysis} and ALDEx2 for differential expression analysis \cite{fernandes2014unifying}.

There is also a dependance on the p-value for statistical analysis, which may not make sense in microbiome research where the number of variables being compared is far greater than the number of samples. Generally in statistical analysis, it has been found that using p-value based approaches with a 0.05 cut off corresponds to a Bayes factor of 3 to 5 (weak evidence). An estimated 17-25\% of such reported results are expected to be wrong, even without p-hacking \cite{johnson2013revised}. We recommend other approaches such as looking at patterns in effect size as with the NAFLD study, where we found that the effect size of the OTUs relatively increased in one condition tended to increase with the severity of the disease.

Additionally, the use of pipelines make it easy for researchers to attempt to analyse their data without looking at the data raw. We recommend visualizing the data in bar graphs (as in Fig.~\ref{nafld_16s_barplot}), principal components (as in Fig.~\ref{nafld_fig2}), as well as looking at the raw counts throughout the analysis process. This way the research can identify outliers that may not be obvious by conventional analytical techniques (Fig.~\ref{fig7}), correct data formatting errors, and ensure that filters and other data transformations are not removing all of the useful information.

\section{Summary}

In this work we have done some methods development and applied it to a study on the gut microbiome of patients with nonalcoholic fatty liver disease (NAFLD). Specifically, we investigated alternate weightings for the UniFrac distance metric (information and ratio UniFrac), allowing the visualization of outliers in certain cases (Fig.~\ref{fig7}), as well as the spread of similar but non-identical data (Fig.~\ref{fig8}). In the NAFLD study, ratio UniFrac produces a principal component analysis with 34.8\% of the variance explained in the first component, compared to 24.4\% for weighted UniFrac and 14.4\% in unweighted UniFrac.

We have also found that many studies in the field are not performed in a statistically sound way, publishing results that cannot be reproduced. Resources, software, and tool recommendations are made in the previous section to prevent this.

The field of microbiome research is in need of standards, such as those set for clinical genomics. When clinical genomics was a budding field, many genome wide association studies were published claiming to have found single nucleotide polymorphisms (SNPs) corresponding to genetic conditions. Discordant results between similar studies prompted the development of standards to ensure statistical validity in analysis and reproducible results. The quality of this information was paramount as study results moved from research labs to clinics for patient genetic counselling. Factors contributing to irreproducibility included batch effects from sample processing \cite{leek2010tackling}, ancestry differences \cite{price2006principal}, and variations in genotype calling methods \cite{miclaus2010variability}, and recommendations were made to avoid pooling the sequences together \cite{mccarthy2008genome}, and for using sample sizes in the thousands \cite{burton2007genome}, stratification detection \cite{price2006principal}, and technical replicates \cite{hong2012technical}, and experimental validation \cite{mccarthy2008genome}. Patient genomes must be sequenced at 30 times coverage or higher to validate the presence of SNPs \cite{rehm2013acmg}

Interestingly, the field of microbiome research seems to have standardized too early. Efforts such as the Human Microbiome Project, set a precident for the types of analyses performed, as well as the tools and techniques researchers use. Some of these have foundations in ecology and are not necessarily applicable to microbiome research, and only a limited number of alternatives have been discussed in the literature. Pipelines such as QIIME \cite{caporaso2010qiime} and mothur \cite{schloss2009introducing} make it comparatively difficult for researchers to explore other analysis options, both in terms of analysing the data, but also in terms of getting alternative options published, due to bias from peer reviews for the standard techniques.

The field of microbiome research has shown lots of promise, yeilding findings such as an obesity-associated increased capacity for energy harvest \cite{turnbaugh2006obesity}, and leading to clinical interventions for diseases such as \textit{C. diff} \cite{petrof2013stool}. With more time and more research, tools and techniques will be developed to perform robust microbiome research, potentially leading to methods to modulate the microbiome and increase quality of life through preventative and restorative medical interventions.
